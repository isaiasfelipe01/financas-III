\documentclass[a4paper,12pt]{article}

% --- Pacotes Fundamentais ---
\usepackage[utf8]{inputenc}      % Codificação do arquivo (permite acentos)
\usepackage[T1]{fontenc}         % Codificação da fonte
\usepackage[brazil]{babel}       % Idioma português
\usepackage{indentfirst}         % Indenta o primeiro parágrafo de cada seção
\usepackage{geometry}            % Configuração das margens
\geometry{a4paper, left=3cm, right=2cm, top=3cm, bottom=2cm}

% --- Pacote de Citações (Padrão ABNT) ---
% O comando \citeonline faz a citação integrada ao texto: Autor (Ano)
% O comando \cite faz a citação entre parênteses: (AUTOR, Ano)
\usepackage[alf]{abntex2cite}

% --- Informações de Capa ---
\title{\textbf{Introdução}}

\author{
    Isaias Felipe Silva de Sousa \\
    João Pedro Stênio Farias Silva
}

\date{Finanças Aplicadas III}

% --- Início do Documento ---
\begin{document}

\maketitle
\section{Introdução}

Decidir como financiar o crescimento é, talvez, um dos dilemas mais persistentes na gestão financeira. Embora existam diversas teorias, a estrutura de capital permanece um tema central e controverso em finanças corporativas. Nesse contexto, a \textit{Pecking Order Theory} (POT), originalmente proposta por Myers e Majluf (1984), oferece uma perspectiva comportamental interessante: ela sugere que os gestores não perseguem uma meta rígida de dívida, mas sim seguem uma ``linha de menor resistência''. Devido à assimetria de informação — o fato de os gestores saberem mais sobre a saúde da empresa do que o mercado —, existe uma preferência clara por financiar investimentos primeiro com recursos próprios (lucros retidos), evitando o escrutínio e os custos de novas emissões. Quando o dinheiro interno acaba, a preferência recai sobre a dívida, deixando a emissão de novas ações (\textit{equity}) apenas como um último recurso, quase desesperado.

O problema que buscamos investigar neste estudo toca no coração dessa dinâmica: \textbf{no momento de ajustar o financiamento anual, as companhias realmente seguem essa escada hierárquica (lucros retidos > dívida > equity)?} Nossa hipótese central é que sim, existe uma ordem observável na decomposição do déficit de financiamento. Esperamos demonstrar que a emissão de dívida não é aleatória, mas funciona como o principal ``amortecedor'' para cobrir a lacuna financeira quando o fluxo de caixa operacional não é suficiente.

Para testar isso, utilizaremos o modelo clássico de Shyam-Sunder \& Myers (1999), que nos permite verificar a validade da POT ao analisar como o déficit de financiamento impulsiona a variação da dívida. Além disso, não olharemos apenas para os números frios; buscaremos entender como a política de dividendos (\textit{payout}) e as restrições financeiras reais — como o acesso limitado ao crédito em momentos de crise — interagem e, por vezes, forçam desvios dessa teoria.


\end{document}