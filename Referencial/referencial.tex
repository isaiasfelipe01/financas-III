\documentclass[a4paper,12pt]{article}

% --- Pacotes Fundamentais ---
\usepackage[utf8]{inputenc}      % Codificação do arquivo (permite acentos)
\usepackage[T1]{fontenc}         % Codificação da fonte
\usepackage[brazil]{babel}       % Idioma português
\usepackage{indentfirst}         % Indenta o primeiro parágrafo de cada seção
\usepackage{geometry}            % Configuração das margens
\geometry{a4paper, left=3cm, right=2cm, top=3cm, bottom=2cm}

% --- Pacote de Citações (Padrão ABNT) ---
% O comando \citeonline faz a citação integrada ao texto: Autor (Ano)
% O comando \cite faz a citação entre parênteses: (AUTOR, Ano)
\usepackage[alf]{abntex2cite}

% --- Informações de Capa ---
\title{\textbf{Referencial Teórico}}

\author{
    Isaias Felipe Silva de Sousa \\
    João Pedro Stênio Farias Silva
}

\date{Finanças Aplicadas 3 \\ \today}

% --- Início do Documento ---
\begin{document}

\maketitle


\section{Referencial Teórico}

\subsection{Fundamentos da Estrutura de Capital e o Teorema de Modigliani e Miller}
A estrutura de capital refere-se à composição das fontes de financiamento de longo prazo de uma empresa, definindo a proporção entre capital de terceiros (dívida) e capital próprio (patrimônio líquido) utilizada para financiar seus investimentos \cite{assaf2014}. O objetivo central das decisões financeiras é maximizar a riqueza dos acionistas, o que implica buscar uma estrutura que minimize o custo total de capital (WACC) e maximize o valor da empresa \cite{gitman2015, assaf2014}.

O ponto de partida para o debate moderno sobre o tema são as proposições de Modigliani e Miller (MM). Em 1958, sob a hipótese de mercados perfeitos e sem impostos, MM argumentaram que a estrutura de capital é irrelevante para o valor da empresa, pois este seria determinado apenas pela rentabilidade dos ativos e pelo risco do negócio \cite{modigliani1958, assaf2014}. Segundo essa proposição, o valor da firma independe da forma como ela é financiada \cite{assaf2014}.

Posteriormente, ao introduzirem os impostos corporativos no modelo, MM reconheceram que o endividamento poderia aumentar o valor da empresa devido à dedutibilidade fiscal dos juros, sugerindo, em uma situação extrema, que a estrutura ótima seria composta quase integralmente por dívidas \cite{modigliani1963, assaf2014}. Contudo, na prática, as empresas limitam seu endividamento devido aos custos de dificuldades financeiras (\textit{financial distress}) e de agência, buscando um equilíbrio que evite que os custos de falência eliminem os ganhos fiscais da alavancagem \cite{assaf2014}.

\subsection{Teoria do \textit{Trade-off} (TOT)}
A Teoria do \textit{Trade-off} (TOT) contrapõe os benefícios fiscais da dívida aos custos associados à falência e às dificuldades financeiras. Segundo essa teoria, as empresas buscam uma estrutura de capital ótima ("alvo") equilibrando a economia fiscal gerada pelos juros com os custos de falência e de agência que surgem com o aumento da alavancagem \cite{souza2024, brealey2020}.

Os custos de agência emergem dos conflitos de interesse entre acionistas e credores (ou gerentes), que se acentuam com o maior endividamento \cite{jensen1976}. Assim, a TOT prevê que empresas maiores e mais lucrativas, que possuem menor risco de falência e maior capacidade de usufruir dos benefícios fiscais, tenderiam a ser mais endividadas \cite{agyei2020}. Em contrapartida, empresas com alto risco de negócio ou ativos intangíveis deveriam utilizar menos dívida. A TOT sugere uma relação positiva entre lucratividade e endividamento, pois empresas mais rentáveis têm mais lucro tributável para proteger \cite{souza2024}.

\subsection{Teoria de \textit{Pecking Order} (POT)}
A Teoria de \textit{Pecking Order} (POT), ou Teoria da Hierarquia, fundamenta-se na existência de assimetria de informação entre gestores (\textit{insiders}) e investidores externos. Para evitar a subavaliação dos ativos e os custos de emissão de novas ações, os gestores seguem uma ordem de preferência no financiamento \cite{myers1984, myers1984b}:
\begin{enumerate}
    \item \textbf{Recursos Internos:} Prioridade para lucros retidos.
    \item \textbf{Dívida:} Se os recursos internos forem insuficientes, a empresa recorre a empréstimos, começando pelos de menor risco.
    \item \textbf{Ações:} A emissão de novas ações é o último recurso, utilizada apenas quando a capacidade de endividamento se esgota, pois é vista pelo mercado como um sinal negativo (sinalização de que a ação pode estar sobreavaliada) \cite{assaf2014}.
\end{enumerate}

Ao contrário da TOT, a POT sugere que não existe uma estrutura de capital "alvo" definida; o nível de dívida é o resultado acumulado das necessidades de financiamento ao longo do tempo \cite{shyam1999}. Uma previsão central da POT é que empresas mais lucrativas utilizam \textit{menos} dívida, pois conseguem financiar suas operações com o caixa gerado internamente \cite{souza2024, ross2015}.

\subsection{Evidências Empíricas e Comparação entre Teorias}
Estudos recentes buscam testar a aderência das empresas a essas teorias, muitas vezes com resultados mistos ou complementares:

\begin{itemize}
    \item \textbf{Lucratividade:} A relação negativa entre lucratividade e endividamento é um dos indicadores mais fortes da POT. Estudos com pequenas e médias empresas (PMEs) em Gana e na Índia confirmam que empresas mais rentáveis tendem a se endividar menos \cite{agyei2020, singh2020}. No entanto, evidências no Brasil sugerem um comportamento híbrido: a lucratividade relaciona-se positivamente com o endividamento de longo prazo (apoiando a TOT) e negativamente com o de curto prazo (apoiando a POT) \cite{souza2024}.
    
    \item \textbf{Crescimento e Tamanho:} A POT sugere que empresas com maiores oportunidades de crescimento demandam mais recursos externos (dívida) quando o caixa interno é insuficiente. Já a TOT pode prever uma relação negativa, pois o crescimento muitas vezes envolve ativos intangíveis que não servem como garantia \cite{agyei2020}. Evidências no Brasil indicam que oportunidades de crescimento têm relação positiva com a dívida, apoiando a POT \cite{souza2024}. Quanto ao tamanho, empresas maiores tendem a ter mais dívidas devido à menor assimetria de informação e menor volatilidade, apoiando ambas as teorias sob diferentes perspectivas \cite{agyei2020}.
    
    \item \textbf{Decisões de Déficit e Superávit:} Pesquisas indicam que as empresas seguem a \textit{Pecking Order} rigorosamente ao emitir ou resgatar dívidas, mas tendem a desviar desse comportamento ao emitir ou recomprar ações, especialmente em mercados como o dos EUA \cite{kakouris2025}.
\end{itemize}

\subsection{O Contexto de Mercados Emergentes e PMEs}
Em mercados emergentes e para Pequenas e Médias Empresas (PMEs), as imperfeições de mercado, como a assimetria de informação e a escassez de crédito de longo prazo, tornam a POT particularmente relevante.

\begin{itemize}
    \item \textbf{Brasil:} O mercado brasileiro caracteriza-se por altas taxas de juros e escassez de fontes de financiamento de longo prazo privadas, levando as empresas a dependerem de lucros retidos ou dívidas de curto prazo \cite{assaf2014}. A estrutura de capital tende a ser concentrada no curto prazo, o que sacrifica a capacidade de investimento e aumenta o risco de refinanciamento.
    
    \item \textbf{Índia e Turquia:} Estudos com PMEs listadas na Índia indicam que a POT explica melhor as decisões de financiamento, onde a liquidez e a lucratividade são determinantes chave \cite{singh2020}. Similarmente, na Turquia, a POT é válida, e a sensibilidade ao uso de fundos internos e dívida aumenta conforme o nível de investimento cresce \cite{yildirim2021}.
    
    \item \textbf{Gana:} PMEs ganesas também demonstram comportamento alinhado à POT, priorizando fundos internos devido ao subdesenvolvimento do mercado financeiro local \cite{agyei2020}.
\end{itemize}

\subsection{Impacto de Crises (COVID-19) na Estrutura de Capital}
Eventos externos disruptivos, como a pandemia da COVID-19, alteram a dinâmica de financiamento. No Brasil, observou-se um aumento no nível de endividamento das empresas de capital aberto no período pós-pandemia, sugerindo uma necessidade maior de recursos externos para manutenção das atividades frente à crise \cite{souza2024}. Embora a variável pandemia nem sempre apresente significância estatística isolada em modelos de regressão, os impactos na rentabilidade e no valor de mercado influenciaram indiretamente as decisões de estrutura de capital, com empresas buscando liquidez via endividamento. A instabilidade econômica tende a reforçar o comportamento de \textit{Pecking Order}, com empresas recorrendo a dívidas quando a geração interna de caixa é impactada \cite{souza2024}.

\newpage


\bibliography{referencias} % Chama o arquivo referencias.bib

\end{document}