\documentclass[a4paper,12pt]{article}

% --- Pacotes Fundamentais ---
\usepackage[utf8]{inputenc}      % Codificação do arquivo (permite acentos)
\usepackage[T1]{fontenc}         % Codificação da fonte
\usepackage[brazil]{babel}       % Idioma português
\usepackage{indentfirst}         % Indenta o primeiro parágrafo de cada seção
\usepackage{geometry}            % Configuração das margens
\geometry{a4paper, left=3cm, right=2cm, top=3cm, bottom=2cm}

% --- Pacote de Citações (Padrão ABNT) ---
% O comando \citeonline faz a citação integrada ao texto: Autor (Ano)
% O comando \cite faz a citação entre parênteses: (AUTOR, Ano)
\usepackage[alf]{abntex2cite}

% --- Informações de Capa ---
\title{\textbf{A Hierarquia Financeira em Foco:} Uma Análise da Teoria de Pecking Order via Decomposição do Déficit de Financiamento.}

\author{
    Isaias Felipe Silva de Sousa \\
    João Pedro Stênio Farias Silva
}

\date{Finanças Aplicadas III}

% --- Início do Documento ---
\begin{document}

\maketitle

\newpage
\section{Introdução}

Decidir como financiar o crescimento é, talvez, um dos dilemas mais persistentes na gestão financeira. Embora existam diversas teorias, a estrutura de capital permanece um tema central e controverso em finanças corporativas. Nesse contexto, a \textit{Pecking Order Theory} (POT), originalmente proposta por Myers e Majluf (1984), oferece uma perspectiva comportamental interessante: ela sugere que os gestores não perseguem uma meta rígida de dívida, mas sim seguem uma ``linha de menor resistência''. Devido à assimetria de informação — o fato de os gestores saberem mais sobre a saúde da empresa do que o mercado —, existe uma preferência clara por financiar investimentos primeiro com recursos próprios (lucros retidos), evitando o escrutínio e os custos de novas emissões. Quando o dinheiro interno acaba, a preferência recai sobre a dívida, deixando a emissão de novas ações (\textit{equity}) apenas como um último recurso, quase desesperado.

O problema que buscamos investigar neste estudo toca no coração dessa dinâmica: \textbf{no momento de ajustar o financiamento anual, as companhias realmente seguem essa escada hierárquica (lucros retidos > dívida > equity)?} Nossa hipótese central é que sim, existe uma ordem observável na decomposição do déficit de financiamento. Esperamos demonstrar que a emissão de dívida não é aleatória, mas funciona como o principal ``amortecedor'' para cobrir a lacuna financeira quando o fluxo de caixa operacional não é suficiente.

Para testar isso, utilizaremos o modelo clássico de Shyam-Sunder \& Myers (1999), que nos permite verificar a validade da POT ao analisar como o déficit de financiamento impulsiona a variação da dívida. Além disso, não olharemos apenas para os números frios; buscaremos entender como a política de dividendos (\textit{payout}) e as restrições financeiras reais — como o acesso limitado ao crédito em momentos de crise — interagem e, por vezes, forçam desvios dessa teoria.

\section{Referencial Teórico}

\subsection{A Lógica da Pecking Order Theory (POT)}

Diferente da Teoria de \textit{Trade-off}, que imagina o gestor financeiro equilibrando benefícios fiscais e custos de falência para atingir um alvo ideal, a POT postula que a estrutura de capital é o resultado acumulado de decisões passadas baseadas na necessidade imediata de financiar novos projetos. A lógica é guiada pelos custos de seleção adversa: usar o dinheiro que já está no caixa da empresa ``custa'' menos em termos de sinalização ao mercado do que emitir dívida e, certamente, muito menos do que emitir ações, que é visto como o financiamento mais caro e arriscado em termos de informação assimétrica.

No entanto, a literatura recente mostra que a aplicação prática dessa teoria não é preto no branco. Em mercados desenvolvidos, como nos EUA, estudos com empresas pagadoras de dividendos mostram que elas tendem a seguir a POT rigorosamente ao emitir ou pagar dívidas, mas acabam se desviando da teoria quando se trata de emitir ações, muitas vezes influenciadas pelo tamanho da transação necessária \cite{Kakouris2025}.

Já em mercados emergentes, a dinâmica muda. Em países como Gana e Índia, a adesão à POT parece ser ainda mais forte, mas por motivos de sobrevivência e controle. Estudos com PMEs nessas regiões indicam que a rentabilidade e a liquidez têm uma relação negativa com a alavancagem — ou seja, empresas que lucram mais, se endividam menos, preferindo usar seu próprio dinheiro, o que confirma a premissa central da POT \cite{Agyei2020, singh2025pecking}. Isso sugere que, onde o mercado de capitais é menos acessível, a hierarquia de preferências é seguida quase por necessidade.

\subsection{O Papel do Déficit de Financiamento e as Restrições}

O conceito de ``Déficit de Financiamento'' é a peça-chave do modelo de \citeonline{shyam1999testing} que aplicaremos. Basicamente, o déficit ocorre quando o fluxo de caixa operacional da empresa não consegue cobrir seus compromissos de investimentos e dividendos. A teoria prevê que esse ``buraco'' deve ser preenchido, dólar por dólar (ou real por real), com nova dívida.

Contudo, a literatura nos alerta que essa relação não é automática e depende das restrições que a empresa enfrenta:

\begin{itemize}
    \item \textbf{Tamanho e Acesso ao Crédito:} Evidências da Turquia mostram uma nuance interessante: embora a POT seja válida em geral, empresas que já possuem alta alavancagem (ou seja, estão ``estranguladas'' em dívidas) tendem a quebrar a hierarquia. Quando precisam de grandes investimentos, elas são forçadas a emitir ações, pois não conseguem mais crédito, desviando-se da teoria clássica \cite{Yildirim2021}.
    
    \item \textbf{O Contexto Brasileiro e Crises:} No Brasil, a realidade impõe seus próprios desafios. Estudos recentes analisando o período da pandemia de COVID-19 mostram que as empresas aumentaram seu endividamento para sobreviver. Curiosamente, a POT explicou melhor as decisões de curto prazo (imediato), enquanto a teoria de \textit{Trade-off} pareceu guiar as decisões de longo prazo, sugerindo que em momentos de crise, a ``sobrevivência'' segue a hierarquia, mas o planejamento busca o equilíbrio \cite{Souza2024}.
    
    \item \textbf{A ``Hierarquia do Calote'':} Além da captação, há também uma perspectiva comportamental sobre o pagamento. Pesquisas recentes introduzem a ideia de uma \textit{Default Pecking Order} (Hierarquia de Calote), onde empresas em dificuldades escolhem estrategicamente quais credores pagar primeiro (baseado em critérios como patrimônio ou relacionamento), o que reforça a ideia de que a gestão da dívida é altamente racional e hierarquizada \cite{Alexandre2023}.
\end{itemize}

\bibliography{referencias}
\end{document}