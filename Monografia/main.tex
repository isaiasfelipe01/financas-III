\documentclass[a4paper,12pt]{article}

% --- Pacotes Fundamentais ---
\usepackage[utf8]{inputenc}      % Codificação do arquivo (permite acentos)
\usepackage[T1]{fontenc}         % Codificação da fonte
\usepackage[brazil]{babel}       % Idioma português
\usepackage{indentfirst}         % Indenta o primeiro parágrafo de cada seção
\usepackage{geometry}            % Configuração das margens
\geometry{a4paper, left=3cm, right=2cm, top=3cm, bottom=2cm}

% --- Pacote de Citações (Padrão ABNT) ---
% O comando \citeonline faz a citação integrada ao texto: Autor (Ano)
% O comando \cite faz a citação entre parênteses: (AUTOR, Ano)
\usepackage[alf]{abntex2cite}

% --- Informações de Capa ---
\title{\textbf{A Hierarquia Financeira em Foco:} Uma Análise da Teoria de Pecking Order via Decomposição do Déficit de Financiamento.}

\author{
    Isaias Felipe Silva de Sousa \\
    João Pedro Stênio Farias Silva
}

\date{Finanças Aplicadas III}

% --- Início do Documento ---
\begin{document}

\maketitle

\newpage
\section{Introdução}

A decisão de como financiar o crescimento corporativo e os investimentos constitui um dos dilemas centrais na gestão financeira, com implicações diretas sobre a criação de valor, o gerenciamento de risco e o custo de capital das organizações \cite{AssafNeto2014,Damodaran2014}. Desde as proposições fundamentais de Modigliani e Miller (1958, 1963), que estabeleceram a irrelevância da estrutura de capital em mercados perfeitos e a posterior relevância dos benefícios fiscais da dívida, a literatura financeira tem buscado decifrar os determinantes das escolhas de financiamento \cite{Agyei2020,Souza2024}. Nesse cenário, a Teoria de Pecking Order (POT), ou Teoria da Ordem Hierárquica, consolidou-se como uma das abordagens dominantes para explicar o comportamento financeiro das firmas, contrapondo-se frequentemente à Teoria de Trade-Off \cite{Gitman2015,Yildirim2021}.

Fundamentada na premissa da assimetria de informações entre gestores (insiders) e investidores externos (outsiders), a POT postula que não existe uma meta rígida de alavancagem ótima \cite{BerkDeMarzo2017, singh2025pecking}. Em vez disso, as decisões de financiamento seguem uma hierarquia destinada a minimizar os custos de seleção adversa e sinalização negativa ao mercado \cite{Gitman2015,Souza2024}. Segundo esta teoria, as empresas preferem financiar seus investimentos primeiramente com recursos internos (lucros retidos); esgotada essa fonte, recorrem à dívida e, somente em última instância, à emissão de novas ações (equity), uma vez que esta última sinaliza uma possível superavaliação dos ativos da empresa e envolve custos de transação mais elevados \cite{Damodaran2014}.

Apesar da lógica intuitiva da POT, a literatura empírica apresenta resultados mistos e inconclusivos, criando uma lacuna de pesquisa persistente sobre a aplicabilidade universal do modelo em diferentes contextos econômicos e condições financeiras \cite{singh2025pecking, Yildirim2021}. Enquanto o estudo seminal de Shyam-Sunder e Myers (1999) encontrou forte aderência ao modelo em grandes empresas norte-americanas, pesquisas em mercados emergentes e com diferentes restrições financeiras oferecem evidências divergentes. Por exemplo, Agyei et al. (2020) e Singh et al. (2020) encontraram suporte para a POT em empresas de Gana e Índia, respectivamente, destacando a influência da liquidez e lucratividade \cite{Agyei2020, singh2025pecking}. Em contrapartida, estudos recentes como o de Kakouris e Psychoyios (2025) indicam que, embora empresas sigam a ordem hierárquica para emissão e redenção de dívida, elas frequentemente se desviam do modelo ao emitir ou recomprar equity, sugerindo que o comportamento não é uniforme em todas as atividades de financiamento \cite{Kakouris2025}.

O presente estudo visa preencher essa lacuna ao investigar a robustez da hierarquia de financiamento especificamente através da decomposição do déficit de financiamento (financing deficit decomposition), testando como o ajuste anual responde a restrições e políticas de dividendos. A pergunta de pesquisa que norteia esta investigação é: No ajuste do financiamento anual, as companhias preferem lucros retidos, depois dívida, e por fim equity? A relevância desta análise é reforçada pela observação de que empresas em mercados emergentes, como o Brasil, enfrentam custos de capital elevados e restrições de crédito de longo prazo, o que pode exacerbar ou distorcer a aplicação da hierarquia padrão prevista pela teoria \cite{Souza2024, AssafNeto2014}.

Para responder a essa questão, adotamos o modelo de Shyam-Sunder e Myers (1999), analisando as variáveis de déficit de financiamento, variação de dívida/equity, payout e restrições financeiras. A hipótese central de que existe uma ordem hierárquica observável é desdobrada em três hipóteses testáveis:

\begin{itemize}
    \item H1: O coeficiente do déficit de financiamento sobre a variação da Dívida ($\theta$ Dívida) é positivo e próximo de 1.
    \begin{itemize}
        \item Justificativa: Conforme a versão estrita da POT, qualquer déficit de financiamento (investimentos e dividendos não cobertos pelo fluxo de caixa operacional) deve ser coberto marginalmente por nova dívida, evitando a emissão de ações \cite{Kakouris2025, Souza2024}. Resultados recentes em mercados emergentes, como na Turquia, sugerem que a sensibilidade à dívida aumenta conforme o nível de investimento cresce \cite{Yildirim2021}.
    \end{itemize}
    \item H2: Um payout mais alto aumenta o déficit de financiamento e eleva a dependência de dívida.
    \begin{itemize}
        \item Justificativa: A teoria sugere que os dividendos são rígidos (sticky) e as empresas evitam cortes abruptos para não enviar sinais negativos ao mercado\cite{Damodaran2014, Hull2016}. Assim, pagamentos de dividendos reduzem a retenção de lucros, forçando a empresa a recorrer ao próximo nível da hierarquia: a dívida \cite{Souza2024}.
    \end{itemize}
    \item H3: Sob maior restrição financeira, a relação déficit-dívida enfraquece.
    \begin{itemize}
        \item Justificativa: Embora a POT preveja a preferência por dívida, empresas com capacidade de endividamento esgotada (debt capacity constraints) ou alta alavancagem podem ser forçadas a desviar da hierarquia padrão, emitindo equity ou restringindo investimentos para evitar custos de dificuldades financeiras (distress costs) \cite{AssafNeto2014,Kakouris2025}. Evidências indicam que empresas com grandes déficits tendem a apresentar menor aderência à emissão de dívida pura \cite{Kakouris2025}.
    \end{itemize}
\end{itemize}

A análise dessas hipóteses contribuirá para o entendimento de como fatores como a rigidez da política de dividendos e as restrições de capacidade de endividamento moderam a aplicação da Teoria de Pecking Order no ajuste anual da estrutura de capital.

\newpage

\bibliography{referencias}
\end{document}