\documentclass[a4paper,12pt]{article}

% --- Pacotes Fundamentais ---
\usepackage[utf8]{inputenc}      % Codificação do arquivo (permite acentos)
\usepackage[T1]{fontenc}         % Codificação da fonte
\usepackage[brazil]{babel}       % Idioma português
\usepackage{indentfirst}         % Indenta o primeiro parágrafo de cada seção
\usepackage{geometry}            % Configuração das margens
\geometry{a4paper, left=3cm, right=2cm, top=3cm, bottom=2cm}

% --- Pacote de Citações (Padrão ABNT) ---
% O comando \citeonline faz a citação integrada ao texto: Autor (Ano)
% O comando \cite faz a citação entre parênteses: (AUTOR, Ano)
\usepackage[alf]{abntex2cite}

% --- Informações de Capa ---
\title{\textbf{A Hierarquia Financeira:} Uma Análise da Teoria de Pecking Order via Decomposição do Déficit de Financiamento.}

\author{
    Isaias Felipe Silva de Sousa \\
    João Pedro Stênio Farias Silva
}

\date{Finanças Aplicadas III}

% --- Início do Documento ---
\begin{document}

\maketitle

\newpage
\section{Introdução}

A decisão de como financiar o crescimento corporativo e os investimentos constitui um dos dilemas centrais na gestão financeira, com implicações diretas sobre a criação de valor, o gerenciamento de risco e o custo de capital das organizações \cite{AssafNeto2014,Damodaran2014}. Desde as proposições fundamentais de Modigliani e Miller (1958, 1963), que estabeleceram a irrelevância da estrutura de capital em mercados perfeitos e a posterior relevância dos benefícios fiscais da dívida, a literatura financeira tem buscado entender os determinantes das escolhas de financiamento \cite{Agyei2020,Souza2024}. Nesse cenário, a Teoria de Pecking Order (POT), consolidou-se como uma abordagem dominante para explicar o comportamento financeiro das firmas, contrapondo-se frequentemente à Teoria do Trade-Off \cite{Gitman2015,Yildirim2021}.

Fundamentada na premissa da assimetria de informações entre gestores e investidores externos, a POT postula que não existe uma meta rígida de alavancagem ótima \cite{BerkDeMarzo2017, singh2025pecking}. Em vez disso, as decisões de financiamento seguem uma hierarquia destinada a minimizar os custos de seleção adversa e sinalização negativa ao mercado \cite{Gitman2015,Souza2024}. Segundo esta teoria, as empresas preferem financiar seus investimentos primeiramente com recursos internos; esgotada essa fonte, recorrema rescursos de terceiros. Sendo esses, empréstimos em um primeiro momento e, somente em última instância, à emissão de novas ações, uma vez que esta última sinaliza uma avaliação dos ativos da empresa e envolve custos de transação mais elevados \cite{Damodaran2014}.

Apesar da lógica intuitiva da POT, a literatura empírica apresenta tanto resultados a favor dessa ordem de prioridade, quanto resultados desfavoráveis a essa ordem, e em certos casos, apresenta também resultados inconclusivos. Dessa forma, criando uma lacuna de pesquisa sobre a aplicabilidade universal do modelo em diferentes contextos econômicos e condições financeiras \cite{singh2025pecking, Yildirim2021}. Enquanto o estudo seminal de Shyam-Sunder e Myers (1999) encontrou forte aderência ao modelo em grandes empresas norte-americanas, pesquisas em mercados emergentes e com diferentes restrições financeiras oferecem evidências divergentes. Os estudos de \citeonline{Agyei2020} e \citeonline{Singh2025} encontraram suporte para a POT em empresas de Gana e Índia, respectivamente, destacando a influência da liquidez e lucratividade \cite{Agyei2020, singh2025pecking}. Em contrapartida, estudos recentes como o de Kakouris e Psychoyios (2025) indicam que, embora empresas sigam a ordem hierárquica para emissão e redenção de dívida, elas frequentemente se desviam do modelo ao emitir ou recomprar ações, sugerindo que o comportamento não é uniforme em todas as atividades de financiamento \cite{Kakouris2025}.

Portanto, o presente estudo visa preencher tal lacuna ao investigar a robustez da hierarquia de financiamento especificamente através da decomposição do déficit de financiamento, testando como o ajuste anual responde a restrições e políticas de dividendos. Logo, a pergunta de pesquisa que norteia esta investigação é: \textbf{No ajuste do financiamento anual, as companhias preferem lucros retidos, depois dívida, e por fim equity?} 

O objetivo geral deste estudo esta atrelado à avaliação da existência de uma ordem hierárquica observável na escolha do método de financiamento das firmas. Para responder objetivo geral proposto, o mesmo será desdobrada em três objetivos especificos:

\begin{itemize}
    \item Analisar a sensibilidade do endividamento em relação ao déficit de financiamento, testando se as variações de capital de terceiros cobrem a lacuna de recursos internos, conforme preconiza a versão estrita da Pecking Order Theory.
    \item Avaliar o impacto da política de payout sobre a estrutura de capital, verificando se a rigidez dos dividendos atua como um impulsionador do déficit de financiamento e, consequentemente, da dependência de dívida.
    \item Investigar o efeito moderador das restrições financeiras e da capacidade de endividamento na hierarquia de fontes de recursos, identificando se altos níveis de alavancagem ou escassez de crédito enfraquecem a relação entre déficit e emissão de dívida.
\end{itemize}

A relevância desta análise é reforçada pela observação de que empresas em mercados emergentes, como o Brasil, enfrentam custos de capital elevados e restrições de crédito de longo prazo, o que pode exacerbar ou distorcer a aplicação da hierarquia padrão prevista pela teoria \cite{Souza2024, AssafNeto2014}. Portanto, o presente trabalho visa contribuir para o entendimento de como fatores como a rigidez da política de dividendos e as restrições de capacidade de endividamento moderam a aplicação da Teoria de Pecking Order no ajuste anual da estrutura de capital.

\newpage

\bibliography{referencias}
\end{document}