\documentclass[10pt, aspectratio=169]{beamer}

% Configurações de idioma e codificação
\usepackage[utf8]{inputenc}
\usepackage[T1]{fontenc}
\usepackage[portuguese]{babel}
\usepackage{booktabs} % Para tabelas mais bonitas
\usepackage{graphicx}

% Tema e Cores
\usetheme{Madrid}
\usecolortheme{whale}
\usefonttheme{professionalfonts}

% Dados da Apresentação
\title[Hierarquia Financeira e POT]{A Hierarquia Financeira em Foco: \\ Uma Análise da Teoria de Pecking Order via Decomposição do Déficit de Financiamento}
\author[Sousa \& Silva]{Isaias Felipe Silva de Sousa \\ João Pedro Stênio Farias Silva}
\institute[Finanças III]{Finanças Aplicadas III}
\date{\today}

% Início do Documento
\begin{document}

% Slide 1: Título
\begin{frame}
    \titlepage
\end{frame}

% Slide 2: Roteiro
\begin{frame}{Roteiro da Apresentação}
    \tableofcontents
\end{frame}

% Seção 1: Introdução
\section{Introdução e Contexto}

% Slide 3: O Dilema Financeiro
\begin{frame}{O Dilema da Estrutura de Capital}
    \begin{itemize}
        \item \textbf{Decisão Central:} Como financiar o crescimento corporativo e investimentos?
        \item \textbf{Impactos:} Criação de valor, gerenciamento de risco e custo de capital (Assaf Neto, 2014; Damodaran, 2014).
        \item \textbf{Evolução Teórica:}
            \begin{itemize}
                \item \textit{Modigliani \& Miller (1958, 1963):} Da irrelevância em mercados perfeitos à relevância dos benefícios fiscais.
                \item \textit{Trade-Off Theory:} Busca por uma estrutura ótima (benefício fiscal vs. custo de falência).
                \item \textbf{Pecking Order Theory (POT):} Foco da nossa análise (Gitman \& Zutter, 2015).
            \end{itemize}
    \end{itemize}
\end{frame}

% Seção 2: A Teoria de Pecking Order
\section{A Teoria de Pecking Order (POT)}

% Slide 4: Fundamentos da POT
\begin{frame}{Fundamentos da Teoria de Pecking Order}
    \begin{block}{Premissa Central}
        Baseada na \textbf{assimetria de informações} entre gestores (\textit{insiders}) e investidores externos (\textit{outsiders}) (Berk \& Demarzo, 2017).
    \end{block}
    
    \vspace{0.5cm}
    
    \begin{itemize}
        \item Não existe uma meta rígida de alavancagem ótima.
        \item As decisões seguem uma \textbf{hierarquia} para minimizar custos de seleção adversa e sinalização negativa.
        \item A emissão de ações (\textit{equity}) é vista como último recurso, pois sinaliza possível superavaliação dos ativos (Damodaran, 2014).
    \end{itemize}
\end{frame}

% Slide 5: A Hierarquia de Financiamento
\begin{frame}{A Hierarquia de Preferências}
    Segundo a POT, a ordem de preferência para financiamento é:
    
    \begin{enumerate}
        \item \textbf{Recursos Internos:} Lucros retidos (menor custo de informação).
        \item \textbf{Dívida:} Quando os recursos internos esgotam. Fonte segura que evita diluição.
        \item \textbf{Emissão de Ações (Equity):} Última instância. Envolve custos de transação elevados e sinalização negativa ao mercado.
    \end{enumerate}
    
    \vspace{0.5cm}
    \begin{center}
        \textit{"Empresas preferem financiar investimentos com lucros retidos, depois dívida, e por fim equity."}
    \end{center}
\end{frame}

% Seção 3: O Problema de Pesquisa
\section{Lacuna e Problema de Pesquisa}

% Slide 6: Evidências Empíricas Mistas
\begin{frame}{O Debate na Literatura}
    Apesar da lógica intuitiva, os resultados empíricos são inconclusivos:
    \begin{itemize}
        \item \textbf{A favor:} Shyam-Sunder e Myers (1999) em grandes empresas dos EUA; Agyei et al. (2020) em Gana; Singh et al. (2025) na Índia.
        \item \textbf{Contra/Nuances:} Kakouris e Psychoyios (2025) indicam desvios ao emitir ou recomprar \textit{equity}.
        \item \textbf{Fatores Moderadores:} Liquidez, lucratividade e restrições financeiras influenciam a aderência ao modelo.
    \end{itemize}
\end{frame}

% Slide 7: Objetivo e Pergunta
\begin{frame}{Objetivo do Estudo}
    \begin{alertblock}{Objetivo}
        Investigar a robustez da hierarquia de financiamento através da \textbf{decomposição do déficit de financiamento}, testando ajustes anuais frente a restrições e políticas de dividendos.
    \end{alertblock}

    \vspace{0.5cm}
    
    \textbf{Pergunta de Pesquisa:}
    \begin{quote}
        "No ajuste do financiamento anual, as companhias preferem lucros retidos, depois dívida, e por fim equity?"
    \end{quote}
\end{frame}

% Seção 4: Metodologia e Hipóteses
\section{Hipóteses do Estudo}

% Slide 8: Metodologia Base
\begin{frame}{Metodologia: O Modelo de Déficit}
    Adota-se o modelo de \textbf{Shyam-Sunder e Myers (1999)}.
    
    \vspace{0.3cm}
    \textbf{Variáveis Analisadas:}
    \begin{itemize}
        \item Déficit de Financiamento (Investimentos + Dividendos - Fluxo de Caixa Operacional).
        \item Variação de Dívida vs. Variação de Equity.
        \item Payout (Política de Dividendos).
        \item Restrições Financeiras.
    \end{itemize}
    
    \vspace{0.3cm}
    A hipótese central de ordem hierárquica é desdobrada em três hipóteses testáveis.
\end{frame}

% Slide 9: Hipótese 1
\begin{frame}{H1: Aderência Estrita ao Modelo}
    \begin{block}{Hipótese 1}
        O coeficiente do déficit de financiamento sobre a variação da Dívida ($\Delta$ Dívida) é \textbf{positivo e próximo de 1}.
    \end{block}
    
    \textbf{Justificativa:}
    \begin{itemize}
        \item Na versão estrita da POT, qualquer déficit não coberto pelo fluxo de caixa deve ser financiado marginalmente por \textbf{nova dívida}.
        \item Evita-se a emissão de ações a todo custo (Kakouris; Psychoyios, 2025).
        \item Evidências em mercados emergentes sugerem que a sensibilidade à dívida aumenta conforme o investimento cresce.
    \end{itemize}
\end{frame}

% Slide 10: Hipótese 2
\begin{frame}{H2: O Papel dos Dividendos}
    \begin{block}{Hipótese 2}
        Um \textit{payout} mais alto aumenta o déficit de financiamento e eleva a dependência de dívida.
    \end{block}
    
    \textbf{Justificativa:}
    \begin{itemize}
        \item Dividendos são rígidos (\textit{sticky}); empresas evitam cortes abruptos para não enviar sinais negativos (Hull, 2016).
        \item O pagamento de dividendos drena a retenção de lucros (fonte primária).
        \item Resultado: A empresa é forçada a recorrer ao próximo nível da hierarquia: a \textbf{dívida}.
    \end{itemize}
\end{frame}

% Slide 11: Hipótese 3
\begin{frame}{H3: Restrições Financeiras}
    \begin{block}{Hipótese 3}
        Sob maior restrição financeira, a relação déficit-dívida enfraquece.
    \end{block}
    
    \textbf{Justificativa:}
    \begin{itemize}
        \item Empresas com capacidade de endividamento esgotada (\textit{debt capacity constraints}) não podem seguir a POT estrita.
        \item Para evitar custos de dificuldades financeiras (\textit{distress costs}), elas desviam da hierarquia.
        \item Solução forçada: Emitir \textit{equity} ou restringir investimentos, reduzindo a aderência à emissão de dívida pura.
    \end{itemize}
\end{frame}

% Seção 5: Conclusão
\section{Relevância e Conclusão}

% Slide 12: Contexto Brasileiro
\begin{frame}{Relevância para o Brasil}
    Por que estudar isso no contexto brasileiro?
    
    \begin{itemize}
        \item \textbf{Custo de Capital:} Empresas enfrentam custos elevados.
        \item \textbf{Crédito:} Restrições de crédito de longo prazo são comuns.
        \item \textbf{Distorções:} Esses fatores podem exacerbar ou distorcer a aplicação da hierarquia padrão prevista pela teoria em mercados desenvolvidos (Souza; Ávila; Prado, 2024).
    \end{itemize}
\end{frame}

% Slide 13: Considerações Finais
\begin{frame}{Considerações Finais}
    Este estudo busca contribuir para o entendimento de como a \textbf{Teoria de Pecking Order} se comporta no ajuste anual da estrutura de capital.
    
    \vspace{0.5cm}
    \textbf{Contribuições esperadas:}
    \begin{itemize}
        \item Validar se a rigidez dos dividendos força o endividamento.
        \item Entender o limite da dívida (restrições) como fator de quebra da hierarquia.
        \item Fornecer evidências sobre o comportamento financeiro em mercados emergentes.
    \end{itemize}
\end{frame}

% Slide 14: Referências
\begin{frame}[allowframebreaks]{Referências Bibliográficas}
    \footnotesize
    \begin{thebibliography}{99}
        \bibitem{Assaf2014} Assaf Neto, A. (2014). \textit{Finanças Corporativas e Valor}. Atlas.
        \bibitem{Berk2017} Berk, J.; Demarzo, P. (2017). \textit{Corporate Finance}. Pearson.
        \bibitem{Damodaran2014} Damodaran, A. (2014). \textit{Applied Corporate Finance}. Wiley.
        \bibitem{Kakouris2025} Kakouris, K.; Psychoyios, D. (2025). Debt, equity, and the pecking order. \textit{Int. Journal of Financial Studies}.
        \bibitem{Shyam1999} Shyam-Sunder, L.; Myers, S. C. (1999). Testing static tradeoff against pecking order models of capital structure.
        \bibitem{Souza2024} Souza, J. C. M. et al. (2024). Decisões de estrutura de capital... \textit{Revista Evidenciação Contábil \& Finanças}.
        \bibitem{Yildirim2021} Yildirim, D.; Çelik, A. K. (2021). Testing the pecking order theory... \textit{Borsa Istanbul Review}.
    \end{thebibliography}
\end{frame}

\end{document}