\documentclass[aspectratio=169]{beamer}

% Configurações de idioma e codificação
\usepackage[utf8]{inputenc}
\usepackage[T1]{fontenc}
\usepackage[brazil]{babel}
\usepackage{booktabs} % Para tabelas estéticas, se necessário

% Tema Minimalista e Robusto
\usetheme{Madrid}
\usecolortheme{beaver} % Esquema de cores cinza/vermelho (sóbrio)
\setbeamertemplate{navigation symbols}{} % Remove os ícones de navegação (minimalismo)
\setbeamertemplate{itemize items}[circle] % Marcadores simples

% Metadados da Apresentação
\title[A Hierarquia Financeira]{A Hierarquia Financeira: Uma Análise da Teoria de Pecking Order via Decomposição do Déficit de Financiamento}
\author[Sousa \& Silva]{Isaias Felipe Silva de Sousa \and João Pedro Stênio Farias Silva}
\institute[]{Finanças Aplicadas III}
\date{\today}

\begin{document}

% Slide 1: Título
\begin{frame}
    \titlepage
\end{frame}

% Slide 2: Introdução e Contexto
\begin{frame}{Introdução: O Dilema do Financiamento}
    \begin{itemize}
        \item \textbf{O Desafio Central:}
        Como financiar o crescimento corporativo e os investimentos?
        \item \textbf{Implicações:}
        Impacto direto na criação de valor, gerenciamento de risco e custo de capital (Assaf Neto, 2014; Damodaran, 2014).
        \item \textbf{Evolução Teórica:}
        \begin{itemize}
            \item \textit{Modigliani \& Miller (1958, 1963):} Da irrelevância da estrutura aos benefícios fiscais.
            \item \textit{Debate Atual:} Teoria do Trade-Off vs. Teoria de Pecking Order (POT).
        \end{itemize}
    \end{itemize}
\end{frame}

% Slide 3: A Teoria de Pecking Order (POT)
\begin{frame}{A Teoria de Pecking Order (POT)}
    \begin{block}{Fundamentação}
        Baseada na assimetria de informações entre gestores e investidores externos. Não assume uma meta rígida de alavancagem ótima (Berk \& DeMarzo, 2017).
    \end{block}
    
    \vspace{0.5cm}
    
    \textbf{A Hierarquia de Preferências:}
    \begin{enumerate}
        \item \textbf{Recursos Internos:} Lucros retidos (Primeira opção).
        \item \textbf{Recursos de Terceiros:} Dívida/Empréstimos.
        \item \textbf{Emissão de Ações (Equity):} Última instância.
    \end{enumerate}
    
    \vspace{0.3cm}
    \footnotesize{\textit{Motivo:} Minimizar custos de seleção adversa e sinalização negativa ao mercado (Gitman \& Zutter, 2015).}
\end{frame}

% Slide 4: O Problema de Pesquisa
\begin{frame}{Lacuna na Literatura e Problema}
    \begin{itemize}
        \item \textbf{Evidências Divergentes:}
        \begin{itemize}
            \item \textit{EUA:} Forte aderência em grandes empresas (Shyam-Sunder \& Myers, 1999).
            \item \textit{Mercados Emergentes (Gana, Índia):} Resultados variam conforme liquidez e lucratividade.
            \item \textit{Desvios:} Empresas frequentemente desviam do modelo ao emitir ou recomprar ações (Kakouris \& Psychoyios, 2025).
        \end{itemize}
        
        \item \textbf{Foco do Estudo:}
        Investigar a robustez da hierarquia via decomposição do déficit de financiamento.
    \end{itemize}
    
    \begin{alertblock}{Pergunta de Pesquisa}
        "No ajuste do financiamento anual, as companhias preferem lucros retidos, depois dívida, e por fim equity?"
    \end{alertblock}
\end{frame}

% Slide 5: Objetivos
\begin{frame}{Objetivos do Estudo}
    \textbf{Objetivo Geral:}
    Avaliar a existência de uma ordem hierárquica observável na escolha do método de financiamento das firmas.
    
    \vspace{0.5cm}
    
    \textbf{Objetivos Específicos:}
    \begin{itemize}
        \item Analisar a sensibilidade do endividamento em relação ao déficit de financiamento (teste da versão estrita da POT).
        \item Avaliar o impacto da política de \textit{payout} (dividendos) sobre a estrutura de capital e dependência de dívida.
        \item Investigar o efeito moderador das restrições financeiras e capacidade de endividamento na hierarquia.
    \end{itemize}
\end{frame}

% Slide 6: Justificativa
\begin{frame}{Relevância: O Contexto de Mercados Emergentes}
    \begin{itemize}
        \item \textbf{Cenário Brasileiro:}
        \begin{itemize}
            \item Custos de capital elevados.
            \item Restrições de crédito de longo prazo (Souza; Ávila; Prado, 2024).
        \end{itemize}
        
        \item \textbf{Contribuição Esperada:}
        \begin{itemize}
            \item Entender se a escassez de crédito ou a rigidez dos dividendos enfraquecem a relação entre déficit e emissão de dívida.
            \item Verificar a aplicabilidade da teoria clássica em contextos de restrição financeira.
        \end{itemize}
    \end{itemize}
\end{frame}

% Slide 7: Referências (Selecionadas)
\begin{frame}{Principais Referências}
    \footnotesize
    \begin{itemize}
        \item Assaf Neto, A. (2014). \textit{Finanças Corporativas e Valor}. Atlas.
        \item Berk, J.; DeMarzo, P. (2017). \textit{Corporate Finance}. Pearson.
        \item Kakouris, K.; Psychoyios, D. (2025). Debt, equity, and the pecking order. \textit{Int. Journal of Financial Studies}.
        \item Modigliani, F.; Miller, M. H. (1958/1963). The Cost of Capital, Corporation Finance and the Theory of Investment.
        \item Shyam-Sunder, L.; Myers, S. C. (1999). Testing static tradeoff against pecking order models of capital structure.
        \item Singh, K. et al. (2025). Pecking order theory... Evidence for listed SMEs in India. \textit{Vision}.
    \end{itemize}
\end{frame}

% Slide Final
\begin{frame}
    \centering
    \Huge \textbf{Obrigado!}
    
    \vspace{1cm}
    
    \large
    Isaias Felipe Silva de Sousa\\
    João Pedro Stênio Farias Silva
\end{frame}

\end{document}